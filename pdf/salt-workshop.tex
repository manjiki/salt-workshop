\documentclass[dvipsnames]{beamer}
\usepackage{listings}
\usepackage{scrextend}
\changefontsizes{10pt}
\mode<presentation>
\usefonttheme{serif}
\usepackage{xcolor}
\usepackage{dirtytalk}
\usepackage{amssymb}
% YAML highlight code from https://tex.stackexchange.com/questions/152829/how-can-i-highlight-yaml-code-in-a-pretty-way-with-listings
\newcommand\YAMLcolonstyle{\color{magenta}\ttfamily\footnotesize}
\newcommand\YAMLkeystyle{\color{JungleGreen}\ttfamily\bfseries}
\newcommand\YAMLvaluestyle{\color{gray}\ttfamily\footnotesize}
\newcommand\YAMLdashstyle{\color{yellow}\ttfamily\footnotesize}
\makeatletter
\newcommand\language@yaml{yaml}
\expandafter\expandafter\expandafter\lstdefinelanguage
\expandafter{\language@yaml}
{
  keywords={true,false,null,y,n},
  keywordstyle=\color{darkgray}\bfseries,
  basicstyle=\YAMLkeystyle\footnotesize,                                 % assuming a key comes first
  sensitive=false,
  comment=[l]{\#},
  morecomment=[s]{/*}{*/},
  commentstyle=\color{NavyBlue}\ttfamily,
  stringstyle=\YAMLvaluestyle\ttfamily,
  moredelim=[l][\color{magenta}]{*},
  moredelim=**[il][\YAMLcolonstyle{:}\YAMLvaluestyle]{:},   % switch to value style at :
  morestring=[b]',
  morestring=[b]",
}
% switch to key style at EOL
\lst@AddToHook{EveryLine}{\ifx\lst@language\language@yaml\YAMLkeystyle\fi}
\makeatother
\newcommand\ProcessThreeDashes{\llap{\color{cyan}\mdseries-{-}-}}
\usecolortheme[named=orange]{structure}
\title{Saltstack Workshop}
\author{e.mouzeli@logicea.net}
\beamertemplateballitem
\begin{document}
\lstset{captionpos=b,escapechar=\&,extendedchars=false,breaklines=true,keepspaces=true,showstringspaces=true}
\frame{\titlepage}
\frame{\tableofcontents[hideothersubsections]}
\section{Basic Packages}
\begin{frame}[fragile]{Basic Packages}
$\circ$ Create file \texttt{/etc/salt/states/default/init.sls}:
\begin{lstlisting}[language=yaml,keepspaces=true]
thebase:
  pkg.installed: 
    - install_recommends: False
    - pkgs:
      - wget:
      - curl:
      - sysstat:
      - ethtool:
      - screen:

openssh-server:
  pkg.installed

an_ssh_bla:
  pkg.installed:
    - name: openssh-server
\end{lstlisting} 
$\circ$ Run: \texttt{salt 'firefly' state.sls defaults}
\end{frame}
\section{Create pillar data}
\small
\begin{frame}[fragile]{Create pillar data}
\noindent
$\circ$ Create file \texttt{/etc/salt/pillars/users/init.sls}
\begin{lstlisting}[language=yaml]
users:
  inara:
    fullname: Inara Serra
    email: inara@serenity.com
    home: /home/inara
    shell: /bin/bash
    groups:
      - root
    pub_keys:
      - ssh-rsa <snip>6Qd9dNjBpkEW0lJt1NXKQo3== 
    enabled: True
\end{lstlisting}
$\circ$ Run: \texttt{salt 'firefly' pillar.data} \newline \newline
\pause
$\circ$ Create file \texttt{/etc/salt/pillars/top.sls}:
\begin{lstlisting}[language=yaml]
base:
  '*':
    - users
\end{lstlisting}
$\circ$ Run again: \texttt{salt 'firefly' pillar.data} \\
$\circ$ Go ahead and create another user
\end{frame}
\normalsize
\begin{frame}[fragile]{Create pillar data}
$\circ$ Create the following \texttt{location} pillar data and include it in \texttt{top.sls}:
\begin{lstlisting}[language=yaml]
location:
  name: firefly
  domain: serenity.com
  dns:
    - 8.8.8.8
    - 8.8.6.6
  root: /opt/serenity
\end{lstlisting}
\pause
$\circ$ Lastly create some \texttt{elasticsearch} pillar data and add it in \texttt{top.sls} only for your host:
\begin{lstlisting}[language=yaml]
elasticsearch:
  config:
    cluster.name: {{salt['grains.get']('elasticsearch:cluster', "kaylee")}}
    node.name: {{salt['grains.get']('fqdn')}}
    node.master: true
    node.data: true
    bootstrap.mlockall: true
    transport.tcp.compress: true
\end{lstlisting}
\end{frame}
\section{Running states - Create users}
\begin{frame}[fragile]{Running states - Create users}
\small
$\circ$ Create and run user state, using the following:
\begin{lstlisting}[language=yaml]

{{user}}:
  user.present:
    - fullname: {{ info['fullname'] }}
    - shell: {{ info['shell']|default("/bin/true") }}
    - home: {{ info['home']|default("/home/%s" % user) }}
    - groups:
      
      - {{ group }}
       &\pause&
  
  ssh_auth:
    - present
    - user: {{ user }}
    - names:
  
      - {{ pub_ssh_key }}
  
    - require:
      - user: {{ user }}
  

\end{lstlisting}
\end{frame}
\normalsize
\begin{frame}[fragile]{Running states - Create users (Notes)}
\begin{itemize}
\item[$\circ$] Salt's state system will first render all pillar data before rendering any *.sls files.
\pause
\item[$\circ$] \texttt{|default("some\_value")} instructs the jinja renderer to apply this value 
if the variable we requested does not exist.
\pause
\item[$\circ$] \texttt{("/home/\%s" \% user)} works as well as 
python's \texttt{"/home/\{0\}".format(user)}
\pause
\item[$\circ$] \texttt{ssh\_auth} is a salt module that will take care of creating an 
\texttt{authorized\_keys} file under a user's \texttt{.ssh/} diretory, 
import any public keys found in the user's pillar data and lastly, apply the 
appropriate permissions \texttt{(600)}.
\end{itemize}
\end{frame}
\section{Running states - Install elasticsearch}
\begin{frame}[fragile]{Running states - Install elasticsearch}
\small
$\circ$ Create the elasticsearch state (\texttt{init.sls}) and use our elasticsearch 
pillar data as config options for \texttt{elasticsearch.yml}
\begin{lstlisting}[language=yaml]
elasticsearch.yml:
  file.managed:
    - name: /etc/elasticsearch/elasticsearch.yml
    - source: salt://elasticsearch/elasticsearch.yml.j2
    - template: jinja
    - context:
      elasticsearch: {{salt['pillar.get']('elasticsearch')}}
    - backup: minion
    - makedirs: True
\end{lstlisting}
\pause
$\circ$ Create the source file salt's templating engine will render to produce 
\texttt{elasticsearch.yml}. Check the file URI in the \texttt{source} parameter
\pause
\begin{lstlisting}[language=yaml]
{# Alternative syntax is elasticsearch['config'] #}

{{config}}: {{value}}

\end{lstlisting}
\pause
$\circ$ We need the file \texttt{elasticsearch.yml} to be present before the package's installation (it's a debian thing).
\end{frame}
\begin{frame}[fragile]{Running states - Install elasticsearch}
$\circ$ Continue working on the elasticsearch state:
\begin{lstlisting}[language=yaml]
elasticsearch:
  pkgrepo.managed:
    - humanname: Elasticsearch Official Debian Repository
    - name: deb http://packages.elasticsearch.org/elasticsearch/1.5/debian stable main
    - key_url: http://packages.elasticsearch.org/GPG-KEY-elasticsearch
    - file: /etc/apt/sources.list.d/elasticsearch.list 
&\pause\vspace{-10pt}&
  pkg:
    - installed
    - hold: True
    - require:
      - pkgrepo: elasticsearch
      - file: elasticsearch.yml
&\pause\vspace{-8pt}&
  service:
    - running
    - enable: True
    - require:
      - pkg: elasticsearch
    - watch:
      - file: elasticsearch*
\end{lstlisting}
\end{frame}
\normalsize
\begin{frame}[fragile]{Running states - Install elasticsearch (Notes)}
\begin{itemize}
\item[$\circ$] Two states are not permited to share the same name unless they call 
different salt modules.
\pause
\item [$\circ$] The \texttt{require} parameter in the \texttt{pkg} state will ensure that the  
\texttt{pkgrepo} state named \say{elasticsearch} and the \texttt{file} state named 
\say{elasticsearch.yml} will run before installing the \say{elasticsearch} package.
\pause
\item [$\circ$] The \texttt{watch} parameter in the \texttt{service} state 
tells salt to restart the service named \say{elasticsearch} if any \texttt{file} 
state who's name matches the expression 
\texttt{elasticsearch*}. In our example, if there are any changes on 
\texttt{elasticsearch.yml}, the elasticsearch service will restart.
\end{itemize}
\end{frame}
\section{Make your elasticsearch state great again}
\begin{frame}[fragile]{Make your elasticsearch state great again}
$\circ$ Elasticsearch requires java, and we want to use Oracle Java. Go on and ensure that 
Oracle java is installed before installing elasticsearch. What you need is:
\begin{lstlisting}[language=bash,basicstyle=\footnotesize\ttfamily]
PPA: deb http://ppa.launchpad.net/webupd8team/java/ubuntu xenial main
Keyid:  EEA14886
keyserver: keyserver.ubuntu.com
Package name: oracle-java8-installer
\end{lstlisting}
\vspace{15pt}
$\circ$ For reasons beyond the scope of this workshop, you must include 
the following salt module in your state:
\begin{lstlisting}[language=yaml]
accept_license:
  debconf.set:
    - data:
        'shared/accepted-oracle-license-v1-1': {'type': 'boolean', 'value': True}
\end{lstlisting}
\end{frame}
\begin{frame}
\center
\includegraphics[scale=0.8]{36555134}
\end{frame}
\end{document}
